\documentclass[10pt,letterpaper]{article}
\include{settings}

\newcommand{\rulemajor}[1]{\section*{#1}}

\begin{document}
\vspace*{0.2in}

\begin{flushleft}
{\Large
\textbf\newline{Ten Quick Tips for Making Things Findable}
}
\newline
\\
{Sarah Lin}\textsuperscript{1,*}
\\
\textbf{1} RStudio, Inc.
\\
\bigskip
* sarah.lin@rstudio.com
\end{flushleft}

\section*{Abstract}

FIXME: abstract

\section*{Introduction}

FIXME: introduction.

\rulemajor{Tip 1: Isolate the findability problem}

When we talk about findability,
it is important to clarify that there are actually three different levels we're concerned with.
There's finding your work on the web,
finding a specific item within a website,
and finding a piece of information within a specific item.
Depending on what content you've created,
you may have all three areas to think about.
While there are fine-tuning techniques appropriate for each,
I do believe that the remaining nine suggestions will improve access to the information your end-user seeks across the board.
What is important in this first tip is to make sure that when you say ``I can't find it''
you recognize that the problem(s) could be at any one level or more.

\rulemajor{Tip 2: Know your users}

Before making any changes to your item(s), site, or underlaying software,
take time to determine who your users are.
Depending on your content you might have only a small, expert audience,
but if your work is publicly available,
you want to be sure to accept the idea that complete novices will find your work.
They should be able to understand and navigate your content as well as a subject expert---or at the very least,
you don't want to make it excruciatingly difficult for them.

\rulemajor{Tip 3: Mimic real world directions}

It's really common for content designers to think that since their audience is highly educated
they will have either an innate understanding of the way you've structured your content
or unlimited patience to persist until they get what they need.
As I learned through over a decade spent with lawyers, internet users are all the same.
They scan, don't read, click on the first close thing they see and give up really, really soon.
Lots of the time they're on their phones or are task-switching---you do not have their full attention
and they're unlikely to really read while they are in search mode.
That said,
it's your job to make it easier for them by ensuring you have put up as many virtual signs as possible.
Things like breadcrumb trails and menus on websites and headings, file paths, and table of contents on documents
give your users a sense of where the information resides
as part of the larger information ecosystem they're currently ``in''
(because we use physical language to describe our internet behavior, even though we don't physically go anywhere.).
Give your users a virtual map with sign posts and directions so they can figure out where they need to go.

\rulemajor{Tip 4: Understand the difference between form and substance}

I'll go into more depth about tagging below,
but it's important to understand that there are terms to describe what your content \emph{is} (format)
and also terms to describe what it is \emph{about} (subject).
Going back to your users,
what subjects are important to them,
and do those topics carry over or change between differences in format?
This is basically a question of combined terms:
are your format terms uniquely matched to topical terms (blog posts are always about news)
or do you have multiple topics in each format (blog posts and tutorials on the same subject)?

\rulemajor{Tip 5: Do not abbreviate}

Don't abbreviate,
or at the very least,
spell out abbreviations or acronyms in all headings and at least the first time in narrative text.
This goes back to the idea of accessibility for all user experience levels:
spelling out acronyms and abbreviations breaks down the exclusivity of language used by a select group and it makes people feel welcome.
Especially with acronyms,
remember that they're often repurposed by different professions or disciplines
and what seems obvious to you is probably not obvious to a number of other people.

\rulemajor{Tip 6: Use tags}

I can't tell you what tags you should or shouldn't use,
but I can explain a bit about types of tagging systems and considerations you should take into account.
Broadly speaking people use `tag' and `index' and `taxonomy' interchangeably,
though to a librarian they are wildly different things.
Outside of information science we can think of two types of tags:
folksonomies and controlled vocabularies.
Folksonomies are like what you see with tags on Flickr:
it's pretty much a free-for-all.
Users tag with whatever terms they like and the list of possible terms is both uncontrolled (no one is watching for synonyms or misspellings)
and crowd-sourced (created by non-experts).
A controlled vocabulary,
on the other hand,
has a defined list of `official' terms that is created and maintained by an `expert' of some sort.
There may or may not be relationships built between terms:
equivalencies (CA=California), broader/narrower terms, and/or replacement (weed USE marijuana).
The good news is that there is probably a place you can find tags in use that are a similar topic or content to yours
that will inspire the tag structure and/or terms you'll want to use.
If you go the folksonomy route,
beware of plural/singular discrepancies in your terms.

\rulemajor{Tip 7: Maximize software utilization}

Most file storage, word processing programs, and website software actually have built-in metadata capabilities,
though they may be hard to find and harder to understand how to leverage.
The worst part may be getting into the habit of adding metadata to your content after creation.
Just like people who end up with piles of photographs with nothing written on the back,
we all have digital mounds of files and content with no format or subject information in the Properties.
The benefits of having metadata are huge, for users and for you.
You'll need to have done some thinking about format and subject,
but if your website software has any metadata functionality (like tagging),
please use it.

\rulemajor{Tip 8: Utilize textual structure}

Textual content is created and aggregated in so many forms using so many different programs that specificity is difficult.
Generally, you want to exploit the features of the textual program to the greatest degree possible:
using Headings in Google Sheets or hyperlinked Table of Contents in Microsoft Word are two examples that come immediately to mind.
If the program you're using has built in options to establish hierarchy or structure to that text, please use it.

\rulemajor{Tip 9: Name carefully}

This tip came to me a few weeks ago when I went looking for a particular cheatsheet.
It turned out that between the heading and the descriptive text the package name and the words ``cheat sheet'' appeared,
but not next to each other.
A Google search netted a ridiculous number of results,
but the current version of that package's cheat sheet was not on the first two pages.
It was only through knowledge I'd recently gained from a coworker
that I remembered the exact file path to that page and could read through until I found the one that was the right match.
Let this be a lesson:
what you call your content is how you should name it
and all the words should be together (imagine someone will put that name in quotes in the search box).
Also,
think about nicknames or shortened versions and make sure they're present in text or tags
so that the content can be discovered by a search engine and a user.

FIXME: \cite{noble2009} for discussion of naming data files.

\rulemajor{Tip 10: Utilize search \emph{and} browsing}

Research into information-seeking behavior shows that
people use a combination of searching and browsing when they're trying to find information.
As they browse a website or document or file,
they put together a mental map of the content they could possibly find and make a search based on that assumption.
They'll further refine their search as they see results,
perhaps navigating into a result or two to further their understanding.
You've probably seen or done something similar with a print book to determine if it's one you want.
Given this behavior,
it's important to provide access to your information through both searching and browsing.
Tags help with searching,
though browsing a list of tags will tell you a lot about the content contained in the resource you're looking at.
Software that makes use of metadata and structural labels enhances search tremendously.

\section*{Conclusion}

FIXME: conclusion

\bibliography{10-findable}

\end{document}
