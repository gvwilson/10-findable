\documentclass[10pt,letterpaper]{article}
\include{settings}

\newcommand{\rulemajor}[1]{\section*{#1}}

\begin{document}
\vspace*{0.2in}

\begin{flushleft}
{\Large
\textbf\newline{Ten Quick Tips for Making Things Findable}
}
\newline
\\
{Sarah~Lin}\textsuperscript{1,*},
{Ibraheem~Ali}\textsuperscript{2}
{Greg~Wilson}\textsuperscript{1}
\\
\textbf{1} RStudio, Inc.
\textbf{2} Louise M. Darling Biomedical Library, University of California, Los Angeles
\\
\bigskip
* sarah.lin@rstudio.com
\end{flushleft}

\section*{Abstract}

Information ecosystems consist of users, context, and content, all three of
which must be addressed to make information findable and usable. Library science
principles are a framework navigating the information ecosystem. They help researchers
improve findability by focusing on the full scope of their users' needs and by leveraging 
structural elements of software that create, store, and access data,
research findings, or academic communications stored locally or shared on the
internet. As scholars evaluate their research communication strategies,
they can use these steps to improve how their research is discovered and communicated.

\section*{Author summary}

Sarah Lin is the Information Architect and Digital Librarian at RStudio, PBC.
Ibraheem Ali is the Sciences Data Librarian at the University of California, Los Angeles.
Greg Wilson works in the education team at RStudio, PBC.

\section*{Introduction}

Researchers have always had to manage information, but the exponential growth of
electronic data has both required and fostered the creation of new ways to do
this \cite{Rosenfeld2015,Hedden2016}. The problem is not just finding particular
information when needed: information may be stored in many formats, exist in
multiple versions, and need to be shared with varied audiences for both research
and teaching.

In the last 4 years, growing adoption of FAIR principles \cite{Wilkinson2016} has helped
researchers manage the data and other digital objects associated with their professional
work. The FAIR principles encourage data to be: Findable, Accessible,
Interoperable, and Reproducible. Indeed, research reproducibility hinges
on the other 3 principles, which align completely with the following 10 tips. 
This paper is concerned with findability both including research data and extends beyond
it to encompass all types of work products produced during a researcher's career, 
inside or outside academic settings.

Library science offers ways to work through the maze of information generated in
professional life, and librarians' skills can be applied by any researcher who
is looking to resove issues with information organization. The ten quick tips in 
this paper build on the fact that all information ecosystems have users, context, 
and content \cite{Rosenfeld2015}. To solve the information retrieval problem, 
researchers must therefore think broadly about who needs that information and 
the context within which it is created as well as its actual content.

\rulemajor{1. Design for a wide range of users.}

The first step in making information findable is to determine who will be doing
the finding. This includes everyone who might learn from your work, contribute to it,
expand upon it, or re-share information through their own networks \cite{Covert2014}.
While you might think there are only a few relevant experts who know your field well, 
novices and trainees will also need to use your work as they gain experience in the field,
thereby making your actual user base considerably larger and more diverse. Furthermore,
some users need to access scholarship through an intermediary, such as translation
software or a screen reader, as well. Being mindful of all potential users and how they
might need to interact with you and your work is the foundation of all ten tips.

The information you wish to convey and the way it is currently organized may
make perfect sense to you, but its meaning for your users is determined by what
\emph{they} interpret from the information they encounter and the way it's
arranged. This means that the organizational structures you employ are a
communication channel in their own right. To illustrate this, Borges created a
classification of animals whose categories included ``those belonging to the
Emperor,'' ``embalmed ones,'' ``suckling pigs,'' ``those included in this
classification,'' ``those drawn with a very fine camel hair brush,'' and ``those
that look like flies from far away'' \cite{Borges2000}. While this was
deliberately ridiculous, it illustrates the fact that every way of organizing
knowledge embodies choices by the organizer, which may or may not align with
those of the audience. Therefore when preparing materials for sharing it is important
to establish context, to use clear and concise language, and minimize the use of jargon.

More concretely, consider the website of a faculty member coming up for tenure:
She created the website as post-doc to publicize her papers and to make it
easier to fill out grant applications by listing professional activities in one
place, but the site might be useful in other cases as well:

\begin{itemize}

\item
  Colleagues may come to the site looking for un-paywalled copies of her
  papers, to find out what she's currently working on, or where she is next
  going to present her work.

\item
  Tenure committee members might review her accomplishments to determine her
  work's impact.

\item
  A librarian (or a program written by a librarian) might scrape that site for
  journal articles to include in the university's institutional repository.

\item
  A student might come to the site looking for course information or materials.

\end{itemize}

Reaching out to a variety of users with distinct needs to ask for findability feedback will
help you discover any gaps in organizational alignment.

\rulemajor{2. Design with the end in mind.}

Given the current state of technology it can be easy to rapidly create digital
information. With the plethora of software and formats you may employ, you almost
certainly have information in lots of different formats, file types and locations.
The second step in making things findable is to think ahead about the things you
would want to be found at the completion of the project before cataloging what you
have. That way you can have time to anticipate what can go where, and adapt if
necessary before materials are published. This is particularly relevant when
archiving data. Different disciplines operate with their own distinct set of
accepted standards of how things should be archived. One example of this can be
found in the field of structural biology. Nearly all protein or molecular
structures are submitted to the Protein Data Bank (PDB) which creates citable
identifers for each submission and follows a set of standard file types to enable 
reuse. This makes protein structures easily searchable with their identifier 
and easily reusable since files are uploaded in a standardized and widely-used 
format\cite{Wilkinson2016}. Following the standards set by experts in the field, 
will reduce the barrier to finding relevant materials associated with your work.

Figuring out what ``done'' will look like can be personally motivating, but once
again, you must be mindful of what your users will consider a useful outcome, which
may not align with what you would do if you were the information's only consumer.
Remember that your future self is also one of your users: everyone is prone to 
forgetfulness, so anything you do for others will likely pay off for yourself
eventually\cite{Briney2015}.

Remember too that the strategies users employ to find something can look different 
in different contexts. Users may need to find your (published or unpublished)
work on the web, find a specific item within a website, and/or find a particular 
piece of information within a specific file or item within a webpage. Depending on 
your content, you may have challenges in all three areas; the remaining tips on 
context and content will help you address these issues.

Elaborating on the example from the previous tip about the faculty member's website 
we can catch a glimpse of the variety of user needs:

\begin{itemize}

\item
  Tenure committee members will want human-readable descriptions of related sets
  of papers along with links to presentations or posters that discuss them.
 
\item
  A librarian might want bibliographic information for your publications in
  machine-readable form (e.g., BibTeX, MARC, or MODS), either for individual
  items or in bulk for addition to an institutional repository.

\item
  Collaborators might links to articles, plus pointers to the protocols, 
  software and data used in order to reproduce particular results or begin
  their own lineage of inquiry.

\item
  Students might look for prominent links to the university's learning management
  system (LMS) to access course materials, general information about the subject 
  of the faculty member's research group, or employment opportunities. 
  
  \item
  Everyone will want persistent links like Digital Object Identifiers (DOIs) \cite{DOI2020} 
  and ORCID \cite{ORCID2020} identifiers, in order to find your papers or research products.

\end{itemize}

\noindent
Each of the users might want content organized chronologically or topically.
In websites that are generated programmatically using tools such as Blogdown\cite{Xie2017} 
or Wordpress \cite{Williams2015} these organization features may be simple to implement, 
but increasing the number of site navigation options may also make the website confusing 
to navigate and make it unclear how the things they find are related to each other. 
Together, authors must strike a balance between offering filtering and searching options, 
and being as inclusive as possible when deciding what materials are worth sharing upon 
the completed version of their work.

\rulemajor{3. Use textual structure.}

Findability at the document, post, or article level can be improved by taking
advantage of the textual structures that information management programs provide
\cite{Hedden2016}. <might be useful to define textual strctures here>. For example, 
a key part of searching the web is scanning the
text returned by search engines to see if it contains target information.
Textual structure helps that process \cite{Krug2014}: formatted headers (rather
than just enlarged text), bulleted or numbered lists, and \textbf{highlighting}
terms that are important all make both the information and its structure easier
to understand. Similarly, headings and table of contents can be hyperlinked,
which supports both scanning and navigation.
Textual structure aids navigation both by helping users create a mental map of the 
webpage or document they have found, but also by exposing elements utilized by 
screen readers to make your work accessible.

Textual content is created and aggregated in so many forms, using so many
different programs, that it is difficult to specify strategies beyond headings,
lists, and highlighting. However, specialists working in the same field tend to
adopt the same tools, so it is worth exploring how your peers annotate
information as well as creating, manipulating, or storing it. For example:

\begin{itemize}

\item
  GitHub allows users to add tags to issues and commit messages which can then
  be searched across projects.

\item
  Electronic lab notebooks can use XML schemas like Darwin Core, EML, or FITS
  \cite{Briney2015}.

\item
  Using specific Google Docs heading levels creates a table of contents in real
  time, visible when the file is open.

\item
  CSV files do not have a standard way to store metadata, but authors commonly
  created a README or MANIFEST file that describes the structure and content of
  the files in a collection. (See \cite{Pudding} for examples.)

\end{itemize}

On a practical level, templates for file creation, data collection, and lab
notebooks makes it easier to be consistent and to spot inconsistencies.

\rulemajor{4. Add metadata.}

Just like people who end up with piles of photographs with nothing written on
the back, we all have digital mounds of files and content with no metadata
describing when it was created or what it contains. Even the most basic metadata
provides extra clues for information retrieval; however, what you can add
depends on the software you use to create, store, and access your information,
and on the file formats that information is stored in.

Almost all modern operating systems allow you to add information to the
Properties of a file or directory. Databases, word processors, and website
construction programs also have built-in metadata capabilities, though they may
be hard to find and harder to understand how to leverage. To make matters worse,
the fact that metadata is often software-specific makes it easy for
inconsistencies to creep in. For example:

\begin{itemize}

\item
  The tags used on a WordPress website may not be in step with the properties in
  the images on that site.

\item
  Keywords added to a journal article when submitting to the publisher's site
  are not automatically added to the metadata in the PDF being submitted.

\item
  When a citation is copied from a database or repository to a bibliography manager,
  the software may not copy over the structural information implied by the
  article's location in the database.

\end{itemize}

The most difficult thing about metadata, however, is getting into the habit of
creating it in the first place. If you get to choose what software to use, it
helps to pick one that makes simple things simple. For example, most website
generators allow you to type tags into an article's header without having to
define them first. This can lead to a proliferation of synonymous (or
misspelled) tags, but some occasional cleanup is better than tackling a mountain
of untagged information. Repositories that force metadata creation upon submission 
greatly assist efforts to make work product findable, and researchers would be 
well served to replicate those metadata elements within their own file storage 
schemas. Indeed, creating an internal taxonomy (list of terms) or ontology (list of
relationships) at the beginning of a project can make assigning metadata much easier.

You should also examine how metadata can be transferred from an old system to a
new one if you have the luxury of switching software (or have had a change
forced on you). Some form of XML is usually the best option when doing this: it
is likely to be with us for many years to come, and the same pedantry that makes
it tedious for human beings to type and read ensures that programs can read it
without having to guess what its creators actually intended.
FAIR principles help ease the burden of software and storage migration, by encouraging researchers to plan for interoperability in software and data storage from the beginning of their projects, concerns about data migration are obviated.

\rulemajor{5. Use search \emph{and} browsing}

Research on information seeking shows that people search \emph{and} browse when
they're trying to find information. As they browse a website or document or
file, they build a mental map of the content they could possibly find, then
search based on that map. ``In the process, they modify their information
requests as they learn more about what they need and what information is
available from the system'' \cite{Rosenfeld2015}. You have probably seen or done
something similar with a print book, trying to determine if it's one you want by
looking at the table of contents and the back cover. These two functions work
together because search allows users to find information they know they need,
whereas browsing allows users to find information they don't know that they 
need\cite{Bates2002}. Designing for both browsing and searching is especially 
pertinent in a search algorithm environment that often dynamically creates 
results unique to each individual search executed, influenced by the user's 
previous interaction with a particular website and/or internet browser.

You should therefore make information accessible both ways and make it easy to
move from searching to browsing and back again. Tags and other metadata help
with searching, while structural clues tell users about the content contained in
the information they are looking at. That communication, ``enables the answers
to users' questions to rise to the surface and answer questions like, Where am
I? What's here? Where can I go from here?''  \cite{Rosenfeld2015}. Similarly,
``{\ldots}the words you use in the navigation systems and headings of [your
  content] help you find what you're looking \emph{for}, but they also help you
understand what you're looking \emph{at}'' \cite{Arango2018}.

For example, when users don't know exactly what they need, the terms in a menu
help them understand the vocabulary used in this domain and the boundaries of
what is included (i.e., terms are listed) or excluded (i.e., no menu terms
exist). At the same time, the headings in the documents they find act as topical
markers: they help users summarize the information contained in the document,
but also refine what they would search for based on the terms used in those
headings. Navigation bars on websites function in a similar way: if the user
knows exactly what they are looking for, they can scan the menu and select the
option that matches their need.

\rulemajor{6. Mimic real world directions.}

The language we use in digital environments mirrors that used for physical
directions: we ``visit'' or ``go to'' a website without actually changing our
physical location. Using the navigational metaphor consistently helps users
build the mental map mentioned in the previous tip. File paths and breadcrumb
trails on websites give users a sense of where the information resides and
suggest new paths they can take \cite{Krug2014}. For example, the URLs of a
website might all include the name of a section of the site, such as
\texttt{/papers/} or \texttt{/blog/}.  
DOIs and ORCID iDs function as the best kind of internet 'signage,' ensuring users never encounter a "webpage not found" error message in the course of retrieving a particular publication.

While much has been written about web usability in general
\cite{Covert2014,NNG2020}, library science focuses on information-seeking
behavior.  For example, we know that users scan but don't read: they click on
the first close thing they see and give up very, very quickly
\cite{Bates2002}. Your markers and directions should therefore be as consistent
as highway signs with regards to appearance, style, and type of
information. Wherever possible (and it's \emph{always} possible), use mechanisms
that users will have become familiar with elsewhere, such as the vertically
nested folders of file browsers or the left-to-right arrangement of breadcrumb
trails.

\rulemajor{7. Use meaningful names.}

The names of files and URLs of webpages are the one piece of metadata you cannot
avoid creating, so always choose ones that are human-readable and that convey
information about what they name, both when navigated to \emph{and} when
returned in search results. Returning again to the faculty member's website, it
would be easy to name a paper \texttt{plos2020.pdf}, but since other people may
also have published papers in PLoS in 2020, a more structured name such as
\texttt{lin-findability-plos-2020.pdf} will both convey more information at a
glance and retain that information after the paper has been downloaded and put
in a folder with dozens of others.

There are many ways to develop a naming schema, largely related to the nature of
the information you create. At the most basic level, ``you should use consistent
names for the same reason that you use good file organization: so you can easily
find and use data later. Additionally, good naming helps you avoid duplicating
information \cite{Briney2015}. Researchers with multiple research projects or
significant complexity in their data sources should establish and document a
unified system of abbreviations for those projects or sources; these can be
summarized in a data dictionary or README file. Consistency is key:
standardizing on lower case, a preferred date format (YYYMMDD or YYYY-MM-DD will
both sort chronologically), and filename suffixes (\texttt{.jpg} instead of
\texttt{.jpeg}) will help everyone find what they need \cite{Wilson2014,Wilson2017}.

Renaming existing files to be consistent with your standards after the fact can
seem like a waste of precious time, but since the research cycle doesn't end
with publication \cite{Briney2015}, there is a very high likelihood that someone
will need to reuse your data and will have to try to figure out what files
corresponded to what part of your research.
Similarly to establishing metadata norms before beginning a project, creating a naming convention that is adopted by any collaborators before research begins will preserve findability into the future.

If you have things to name that are not files, such as projects, web pages, or
document headings, remember that the more generic a term is, the harder it is to
search for: naming a raw data file ``raw'' or a downloaded file ``download''
makes finding the information they contain nearly impossible. A quick test is to
search for the name before adopting it: if dozens of unrelated pages come up,
you may want to pick a different name. You should also think about nicknames or
shortened versions of your names and make sure they are present in text or tags
so that the content can be discovered by a search engine and a user.

\rulemajor{8. Use tags.}

After meaningful names, tags are the easiest and most effective metadata you can
create.  Almost all digital tools allow users to add arbitrary tags to items:
file properties on Windows and labels on GitHub issues are just two examples.
Additionally, almost all search tools leverage tags to narrow a query's
scope. This means that you can now file a single thing in multiple
``locations'', which was not possible in the pre-digital era. Multiple tags also
assist users from varied backgrounds because the terms can be customized to each
type of user your information has.

When choosing tags, be consistent in your depth of topical term assignment (how
specific your terms are) and your selection of terms for subject and format (the
number of terms you use to describe each subject and format). For example, if
you tag some items in an ecological data set with a species name, don't tag
others simply as ``reptile'' unless the species is unknown, in which case you
should:

\begin{itemize}

\item
  tag all items ``reptile'', ``bird'', ``mammal'', and so on for high-level
  searches, and

\item
  tag all items with a species, which might be ``unknown'' or ``NA'' (not
  available), or
  
\item
  tag all items at both general and increasingly specific categories if that is the standard for your discipline \cite{FAIR2020}.

\end{itemize}

What should you tag?  The answer is ``everything'' from informal personal notes
to data sets submitted with publications or included in repositories, because it
is all material you will want to be able to find later.  The benefit of tagging
comes from doing it in all of those situations, not just when a journal
submission requires it.

If you are certain something is for purely personal use, you can create your own
taxonomy of subject keywords, which is called a \emph{folksonomy}. Folksonomies
are what you see with tags on Flickr: early content creators assign terms as
they see fit, and later contributors can use those or add their own. If you take
this route, it's worth reviewing new tags regularly to look for synonyms,
misspellings, differences in capitalization, singular/plural discrepancies, and
other inconsistencies.

What terms you use as tags for personal consumption may not matter much, but
work that is shared with colleagues should use particular terms or tags that
conform to certain standards \cite{FAIR2020}.  These terms typically come from
taxonomies, thesauri, and ontologies: taxonomies and thesauri generally have
built-in subject hierarchies that can help you create navigational structure,
while ontologies map relationships between ideas. Crucially, all three are
\emph{controlled vocabularies}: they are a defined list of terms created and
maintained by experts rather than being crowdsourced like a folksonomy.

There may or may not be relationships built between terms in a controlled
vocabulary, such as equivalencies (``CA'' for ``California''), broader/narrower
terms (United States/California), and/or replacement (weed \emph{use}
marijuana). Established subject terms will match article databases, data
repositories, and library catalogs that you and your users might already be
familiar with, which will again aid search and navigation.  Well-known examples
in the United States include the National Cancer Institute (NCI) Thesaurus
\cite{NCI2020} and the Medical Subject Headings (MeSH) \cite{ASI2020}.

\rulemajor{9. Understand the difference between format and subject.}

However you create tags, you need to address the distinction between format and
subject. Format describes what your content \emph{is}, while subject describes
what it is \emph{about} \cite{Joudrey2015}. About-ness is the most common
content analysis, but is-ness issues will probably affect people's ability to
use your information, so you may want to add metadata to make it explicit.

A simple example of this is a blog post on a researcher's professional website. The post is \emph{about} a
subject, like a book review, but it \emph{is} a blog post rather than biographical details,
bibliography, or a list of currently taught classes.  Going back to your
users, what subjects are important to them? And do those topics carry over or
change between differences in format? For a librarian, this is basically a
question of combined terms: are your format terms uniquely matched to topics
(e.g., blog posts are always about news) or do you have multiple topics in each
format (e.g., blog posts and tutorials on the same subject)?

Similarly, you can rely on filename suffixes to distinguish computational
notebooks from PDF files, tabular data sets, or slide decks, but should use
tagging, a filename convention, or a description in a README to tell people
whether the contents are raw information, tidied-up data, or an aggregation of
several underlying datasets. This enables users to search by topic, format, or
both.

Since dissemination sometimes changes a file's format (e.g., printing slides to
a PDF), naming and metadata conventions tend to be more robust as well as more
informative than relying on file types. Once again, structural clues can help: a
folder specifically for conference presentations may contain one sub-folder for
each presentation, which in turn contains the PowerPoint and PDF versions of the
presentation with exactly the same names but different filetype suffixes.
Likewise, journal articles you store will need a naming or structural convention
to distinguish articles you have written from those you have downloaded for your
own use.

\rulemajor{10. Do not abbrvt.}

Acronyms and abbreviations make communication between those who know them more
efficient at the price of making them less accessible to newcomers. Spelling
out acronyms and abbreviations that you take for granted (or hyperlinking to
their definitions) therefore makes information easier to find and interpret for 
a diverse audeince. When doing this, remember that acronyms are often repurposed by
different professions or disciplines: what seems obvious to you is probably not
obvious to people from other communities. Since every discipline has some common
abbreviations, write them all out in full the first time they appear or create
or point to a term dictionary.

\section*{Conclusion}

Changing work habits is hard, so remember that while perfection isn't possible,
progress is. Start by deciding whether to begin your next project with a new set
of information organizing principles or to go back and alter existing artifacts
\cite{Briney2015}. You might also consider this process as you would a research 
experiment, and incorporate one small change at a time. Whichever you choose, the
``ways you enforce your way of doing things changes how users think about the 
place[s] you made and perhaps ultimately, how they think about you'' \cite{Covert2014}.

\bibliography{10-findable}

\end{document}
